
\documentclass[ structure = article
              , maketitlestyle = standard
              , secstyle = center
              , secfont = roman
              , liststyle = aligned
              ]{suftesi}

\usepackage[no-math]{fontspec}
\usepackage[rm,tt=false]{libertine}
\usepackage[scale=.85]{sourcecodepro}

\usepackage{polyglossia}
\setmainlanguage{italian}

\usepackage[italian=quotes]{csquotes}

\usepackage{hyperref}
\hypersetup{ breaklinks
           , hidelinks
           }

%\usepackage{anyfontsize}
%\usepackage{indentfirst}

\usepackage{booktabs}

% \usepackage{abstract}
%\renewcommand\abstracttextfont\normalfont

%\usepackage[ bibstyle = alphabetic
%           , citestyle = alphabetic
%           , pluralothers=true
%           , autolang=langname
%           ]{biblatex}
%\addbibresource{biblio.bib}

\usepackage{libertinust1math}
\usepackage{MnSymbol}
\usepackage{mathtools}
\let\underbrace\LaTeXunderbrace
\let\overbrace\LaTeXoverbrace
\usepackage[bb=ams]{mathalfa}
\usepackage{amsthm}

\theoremstyle{definition}
\newtheorem{esercizio}{Esercizio}[section]
\newtheorem{richiamo}{Richiamo}[section]

\newenvironment{soluzione}{\begin{proof}[Soluzione]}{\end{proof}}

\usepackage{tcolorbox}
\tcbuselibrary{theorems, breakable, skins}

\tcolorboxenvironment{esercizio}{
  % oversize,
  blank, breakable,
  left = 3mm, top=1mm, bottom=1mm,
  borderline west={1mm}{0pt}{gray}
}

%\usepackage{enumitem}
%\setlist{label=(\alph*)}

\usepackage{listings}
\lstset{basicstyle=\tt}

\renewcommand\hat\widehat
\renewcommand\tilde\widetilde
\renewcommand\bar\overline
\newcommand\set[1]{\left\{#1\right\}}
\newcommand\abs[1]{\left\lvert#1\right\rvert}
\newcommand\inv[1]{#1^{-1}}
\newcommand\ind[2]{\mathbf 1_{#1}\left(#2\right)}
\newcommand\mbb[1]{\mathbb #1}
\newcommand\mcal[1]{\mathcal #1}
\newcommand\att{\mbb E}
\newcommand\var{\operatorname{Var}}
\newcommand\deriv[2]{\frac{\mathrm d #1}{\mathrm d #2}}
\newcommand\pderiv[2]{\frac{\partial^{#1}}{\partial #2^{#1}}}
% integrale di Lebesgue
\newcommand\leb[3]{\int_{#1} #3 \mathrm d #2}
% note interne
\newcommand\nota[1]{\textcolor{red}{#1}}


%%% Local Variables:
%%% mode: LaTeX
%%% TeX-engine: luatex
%%% TeX-master: "main"
%%% End:
