
\section{Statistica Descrittiva}

\begin{esercizio}[Problema 1, scritto del 12/02/2024]
  Un'indagine ISTAT su $n = 100$ famiglie analizza la relazione tra il
  carattere $X$ relativo alla numerosità del nucleo familiare e il
  carattere $Y$ realtivo al numero di locali della rispettiva
  abitazione. L'indagine produce la seguente tabella
  \[
    \begin{array}{c|ccc}
      & X = 1 & X = 2 & X = 3 \\
      \midrule
      Y = 2 & 20    & 10    & 7     \\
      Y = 3 & 9     & 13    & 13    \\
      Y = 4 & 2     & 5     & 21
    \end{array}
  \]
  Si affrontino i seguenti quesiti.
  \begin{enumerate}[leftmargin=*]
  \item Calcolare la media aritmetica (pesata) e la mediana del
    carattere $Y$.
  \item Valutare la variabilità di $Y$ tramite l'uso delle differenza
    media semplice (di ordine $p = 1$).
  \item Valutare la concentrazione del carattere $Y$ rispetto alla
    distribuzione uniforme, esplicitando il grafico della curva di
    Lorentz-Gini e calcolando un opportuno indice di concentrazione.
  \item Calcolare la tabella relativa alla distribuzione di
    probabilità congiunta corrispondente alla situazione (ideale) di
    indipendenza dei due caratteri.
  \item Scegliere un indice di connessione tra $X$ e $Y$ e calcolarlo
    numericamente coi dati forniti.
  \item Calcolare la tabella corrispondente alla distribuzione di
    probabilità congiunta corrispondente alla situazione (ideale) di
    massima concordanza tra i due caratteri.
  \item Scegliere un indice di concordanza tra $X$ e $Y$ e calcolarlo
    numericamente coi dati forniti.
  \end{enumerate}
\end{esercizio}

\begin{soluzione}
  {\color{red} [Da riscrivere. Vedere parte commentata
    nel sorgente \TeX{}.]}
\end{soluzione}

% \begin{soluzione}
%   \begin{enumerate}
%   \item Cerchiamo di interpretare la tabella in termini più
%     astratti. Abbiamo uno spazio campionario che è la popolazione
%     sotto indagine e che chiamiamo $\Omega$. Questo insieme è munito della
%     sigma-algebra $2^\Omega$ e della misura del conteggio
%     $\mu : 2^\Omega \to [0, +\infty]$ che manda ogni singoletto
%     $\set{\omega}$ in $1$. Poi, $X, Y : \Omega \to \mathbb{R}$ sono due variabili aleatorie
%     che prendono solo un insieme finito di valori, $\set{1, 2, 3}$ e
%     $\set{2, 3, 4}$ rispettivamente. In ogni cella della tabella
%     stanno scritti semplicemente i valori
%     \[
%       \mu\left[ X = i, Y = j\right] .
%     \]
%     Possiamo ``normalizzare tutto'', poiché $\mu$ è finita, e
%     considerare la misura di probabilità
%     $\mathbb{P} : 2^\Omega \to [0, 1]$ che manda ogni singoletto
%     $\set{\omega}$ in $\frac 1 n$. La misura indotta
%     \[
%       \pi_Y : \mathbb{B} \mathbb{R} \to [0, 1] \,,\ \pi_Y (E) := \mathbb{P} [ Y \in E ] .
%     \]
%     rappresenta quindi la distribuzione della variabile aleatoria
%     $Y : \Omega \to \mathbb{R}$. Possiamo quindi scrivere la funzione di ripartizione
%     \[
%       F_Y(t) = \mathbb{P} [Y \le t] = \frac{37}{100}\ind{[2, +\infty)}{t} +
%       \frac{35}{100}\ind{[3, +\infty)}{t} + \frac{28}{100}\ind{[4, +\infty)}{t}
%       .
%     \]
%     Ecco la media aritmetica del carattere $Y$ è quindi

% \[
%   \bar Y := \int_{\mathbb{R}} t \mathrm d F_Y(t) = \frac{37}{100} \cdot 2 +
%   \frac{35}{100} \cdot 3 + \frac{28}{100} \cdot 7 = 2.91
% \]
% Calcoliamo anche la mediana del carattere $Y$. Abbiamo già calcolato
% la funzione di ripartizione, non rimane che calcolare anche la sua
% \enquote{inversa}
% \[
%   \inv{F_Y}(s) = \inf \set{t \in \mathbb{R} \mid F(t) \ge s}.
% \]
% La mediana è semplicemente
% \[
%   \inv{F_Y} \left(\frac 12\right) = 3 .
% \]

% \item Richiamiamo questo indice
%   \[
%     \frac{1}{{n \choose 2}} \sum_{1 \le i < j \le n} \abs{Y(\omega_i) - Y(\omega_j)}
%   \]
%   dove $\Omega = \set{\omega_i \mid i = 1, \dots{}, 100}$. La differenza
%   $\abs{2-3} = 1$ occorre $37 \cdot 35$, la differenza $\abs{2-4} = 2$
%   occorre $37 \cdot 28$, la differenza $\abs{3-4} = 1$ occorre
%   $35 \cdot 28$ volte. Quindi ecco eseguito il conto sopra
%   \[
%     \frac{2}{100 \cdot 99} (1 \cdot 37 \cdot 35 + 2 \cdot 37 \cdot 28 + 1 \cdot 35 \cdot 28) \approx
%     0.88 .
%   \]

% \item ...
% \end{enumerate}
% \end{soluzione}


\begin{esercizio}[Problema 2, 12/09/2024]
  Sullo spazio misurabile
  \(\left( \Omega, \mathcal{F} \right) := \left( (1, +\infty), \mathbb{B} (1,+\infty) \right)\)
  si considera la misura di riferimento
  \[
    Q : \mathcal{F} \to [0,1] \,,\ Q(B) := \int_{B} x^{-2} \mathrm d x
  \]
  e la misura di prova
  \[
    P : \mathcal{F} \to [0,1] \,,\ P(B) := \alpha \int_{B} x^{-1-\alpha} \mathrm d x
  \]
  dove \(\alpha \in (0,1)\).
  \begin{enumerate}[leftmargin=*]
  \item Calcolare la mediana di entrambe le misure.

  \item Discutere la mutua variabilità rispetto alla distanza
    \(d(x, y) := \left\lvert x-y \right\rvert\). In particolare,
    confrontare le due differenze medie di ordine 1 sfruttando la {\em
      formula di de Finetti-Paciello} secondo cui per una funzione di
    ripartizione \(F\) con supporto in \((0, +\infty)\) vale l'identità
    \[
      \int_{0}^{+\infty} \int_{0}^{+\infty} \left\lvert x-y \right\rvert \mathrm d
      F(x) \mathrm d F(y) = 2 \int_{0}^{+\infty} F(x) [1-F(x)] \mathrm d x .
    \]

  \item Dopo aver verificato che \(P\) è assolutamente continua
    rispetto a \(Q\), calcolare esplicitamente la curva di
    concentrazione di \(P\) rispetto a \(Q\).

  \item Calcolare l'area di concentrazione e, in particolare, dire se
    aumenta o diminuisce in funzione di \(\alpha\).
  \end{enumerate}
\end{esercizio}

\begin{soluzione}
  \begin{enumerate}
  \item Calcoliamo quindi le funzioni di ripartizione, cioè
    \[
      F_Q : \mathbb{R} \to [0,1] \,,\ F_Q(t) := Q \left\{ \omega \in \Omega \mid \omega < t \right\}
      =
      \begin{cases}
        0 & \text{se } t \le 1 \\
        1-\frac{1}{t} & \text{se } t > 1
      \end{cases}
    \]
    e
    \[
      F_P : \mathbb{R} \to [0,1] \,,\ F_P(t) := P \left\{ \omega \in \Omega \mid \omega < t \right\}
      =
      \begin{cases}
        0 & \text{se } t \le 1 \\
        1-\frac{1}{t^\alpha} & \text{se } t > 1
      \end{cases}
    \]
    Le rispettive mediane sono quindi
    \begin{align*}
      & F_Q^{-1} \left( \frac{1}{2} \right) = \inf \left\{ s \in
        \mathbb{R} \left\mid F_Q (s) \ge \frac{1}{2} \right. \right\} = 2 \\
      & F_P^{-1} \left( \frac{1}{2} \right) = \inf \left\{ s \in
        \mathbb{R} \left\mid F_P (s) \ge \frac{1}{2} \right. \right\} = 2^{\frac{1}{\alpha}} \\
    \end{align*}

  \item {\color{red} [Da rivedere meglio\dots]}

  \item Sia \(E \in \mathcal{F}\) qualsiasi tale che \(Q(E) = 0\). Ne segue che
    \(E\) sia una insieme di misura nulla secondo la misura
    unidimensionale di Lebesgue. Ma allora anche \(P(E) = 0\). Quindi
    per il {\sc Teorema di Radon-Nikodym} esiste unica a meno
    uguaglianza quasi ovunque una funzione
    \(Z : \Omega \to [0, +\infty]\) misurabile tale che
    \[
      P(E) = \int_{E} Z \mathrm d Q \text{ per ogni } E \in \mathcal{F} .
    \]
    Questa \(Z\) è la {\em derivata di Radon-Nikodym} e si scrive
    \(\frac{\mathrm d P}{\mathrm d Q}\). Calcoliamola allora. Se
    indichiamo con \(\lambda\) la misura indotta su
    \((\Omega, \mathcal{F})\) dalla misura di Lebesgue, osserviamo anche che
    \(P\) e \(Q\) sono entrambe assolutamente continue rispetto a
    \(\lambda\). Quindi
    \[
      P(E) = \int_{E} Z \mathrm d Q = \int_{E} Z \frac{\mathrm d Q}{\mathrm
        d \lambda} \mathrm d \lambda .
    \]
    All'ultimo membro sappiamo quanto vale
    \(\frac{\mathrm d Q}{\mathrm d \lambda}\) quasi ovunque: grazie al {\sc
      Teorema di Radon-Nikodym} infatti possiamo dire
    \[
      \frac{\mathrm d Q}{\mathrm d \lambda} (x) = x^{-2} \text{ per quasi
        ogni } x \in \Omega .
    \]
    Ancora nuovamente per il {\sc Teorema di Radon-Nikodym}, applicato
    questa volta alla coppia \(P\) e \(\lambda\), si ha che
    \[
      Z(x) \frac{\mathrm d Q}{\mathrm d \lambda} (x) = \alpha x^{-1-\alpha} \text{ per
        quasi ogni } x \in \Omega .
    \]
    Da cui possiamo concludere che
    \[
      Z(x) = \alpha x^{1-\alpha} \text{ per quasi ogni } x \in \Omega .
    \]
    In realtà, possiamo scegliere senza perdere nulla che \(Z\) di
    definire in questo modo su {\em tutto} \(\Omega\), visto che
    l'integrale di Lebesgue ignora insiemi di misura nulla.

    Calcoliamo la funzione di ripartizione di
    \(Z = \frac{\mathrm d P}{\mathrm d Q}\):
    \begin{align*}
      F(t) &:= Q \left\{ x \in \Omega \mid Z(x) \le t \right\} = \\
           &= Q \left\{ x > 1 \left\mid x \le \left( \frac{t}{\alpha}
             \right)^{\frac{1}{1-\alpha}} \right. \right\}
    \end{align*}
    Qui
    \[
      F(t) =
      \begin{cases}
        0 & \text{se } t \le \alpha \\
        1-\left( \frac{\alpha}{t} \right)^{\frac{1}{1-\alpha}} &  \text{se }
                                                       t > \alpha
      \end{cases} .
    \]
    Ecco l'inversa generalizzata di \(F\): per \(y \in (0, 1)\)
    \[
      F^{-1}(y) = \frac{\alpha}{(1-y)^{1-\alpha}} = \alpha (1-y)^{\alpha-1} .
    \]
    Possiamo finalmente scrivere la {\em funzione di concentrazione}
    \[
      \phi (s) := \int_0^s F^{-1}(y) \mathrm d y = 1 - (1-s)^\alpha .
    \]
    
  \item Abbiamo a questo punto tutto quello che serve per calcolare
    l'{\em area di concentrazione}
    \[
      \int_{0}^{1} [s-\phi(s)] \mathrm d s = \frac{1}{\alpha+1} - \frac12 .
    \]
    In particolare, l'area è decrescente rispetto ad \(\alpha\).  \qedhere
  \end{enumerate}
\end{soluzione}

%%% Local Variables:
%%% mode: LaTeX
%%% TeX-master: "main"
%%% TeX-engine: luatex
%%% ispell-local-dictionary: "italian"
%%% End:
